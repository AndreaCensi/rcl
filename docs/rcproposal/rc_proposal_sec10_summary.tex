
\section{Summary (1 page)}



The last 10 years have seen many progress in autonomous robotics. The key enabler
to autonomy were advances in robotic sensing hardware and the corresponding
new techniques for dealing with the sensors data. For example, the introduction
of the laser range finder as a standard robotic sensor allowed for the first
time to have very reliable localization and mapping, through the introduction
of the probabilistic/Bayesian paradigm (Markov localization, particle filters,
etc.). Those simultaneous advances in sensors and theory enabled autonomous
exploration of large scale environments and ultimately the feasibility of applications
such as autonomous cars. Currently a large amount of research is devoted to
monocular vision, which will be the key technology in the next 5 years, especially
for micro aerial vehicles, whose limitations in payload and the outdoors operations
prevent using clunky and power-hungry sensors like range-finders. 

But, looking farther in the future, what will be the sensor modality of the
future, and what theoretical paradigm needs to change to take advantage of it?
Our view is that \emph{dynamic} \emph{sensors} in the style of the DVS camera
are the most likely answer. The reasoning is simple: the performance of autonomous
robots, measured by agility, in tasks such as exploration mainly depends on
the accuracy of perception. In particular, in a sense-plan-act architecture
what matters is the latency of the pipeline. The latency depends on the frequency
of the sensor data, plus the time it takes to process the data. It is typical
in current robots to have latencies of 200ms or more. This puts a hard bound
on the agility of the platform. A DVS camera virtually eliminates the latency:
data is transmitted using events that have latency in the order of few microseconds.
However, this incremental information is completely different than the data
we have now from traditional sensors, so that a new paradigm is needed to deal
with this data. Moreover, the sensors that so far have been designed for general
purpose use must be adapted to the case of robotics.

This transcontinental interdisciplinary collaborative project is dedicated to
paving the way for widespread use of dynamic sensors in the field of robotics,
from both the theoretical and the practical point of view. The three partners
provide a unique set of competencies. The Institute of Neuroinformatics (Dr.
Delbruck) is a leader in the design and production of neuromorphic dynamic vision
sensors. University of Zurich (Prof. Scaramuzza) is a leader in perception and
control for micro-aerial vehicles, which are the reference platform for the
project. Caltech (Dr. Censi) brings the competencies for dealing with low-level
sensorimotor data and for designing control architectures.

This project is particularly well suited to Sinergia because it is the only
institutional source of funding that allows transcontinental collaborations
with US partners. Moreover, Prof. Scaramuzza and Dr. Censi are young scientists
at the beginning of the research career.


\section{Research plan}




\subsection{Current state of research in the field}

Making reference to the most important publications, particularly by other authors,
please explain: - which previous insights provided the starting point and basis
for the planned studies; - in which areas research is needed, and why; - which
important, relevant research projects are currently underway in Switzerland
and abroad.


\subsection{Current state of your own research}

For a new application, please present the research work that the various (co-)applicants
have already undertaken in the relevant field or in related fields and mention
the corresponding publications.


\subsection{Research plan of the entire project}

Based on the information provided under 2.1 and 2.2, please indicate: - the
benefits expected from a co-operative approach (as a net- work); - the concrete
overall goals that you expect to achieve in the course of the project; 2.3 -
the scientific approach (combination of different methods, tech- niques, etc.)
to be used in addressing the overall goals of the project. If the submitted
project is declared as interdisciplinary, you must explain on approx. one page
what this interdisciplinarity consists of, why it is essential and how it will
be taken into consideration when the research plan is implemented.


\subsection{Organization of the collaboration}

Please answer the following questions: What scientific contribution is each
of the sub-projects expected to make? How will the research work be structured
as regards content and schedule? Which specific measures will be taken to enhance
interaction (joint development of concept, continual exchange of knowledge,
etc.) and to integrate results. Please specify the time and the means you will
devote to this end? How will the collaboration of different groups in a network
promote the education of young scientists (postdocs and doctoral candidates)?


\subsection{Relevance and impact}

Please describe the impacts you expect your research in the proposed project
to have for the discipline and for science as a whole (research and education/teaching).
In addition, please mention the form in which you wish to make your research
results public (arti- cles in science journals, monographs, conference proceedings
etc). If applicable, please indicate whether and to what extent the proposed
project will have a broader impact and what this impact will be.


\section{Subproject 1}

Please compile a research plan for each of the sub-projects involved. The research
plan of a sub-project must not exceed 10 pages and 40,000 characters (with spaces),
illustrations, formulae, tables and bibliographies included. A minimum of point
10 font size and 1.5 line spacing must be used.


\subsection{Interactions with the overall project}

Please explain how this sub-project is linked to the other sub- projects and
to the overall project.


\subsection{Detailed research plan}

Please specify the approach you are taking and the concrete objec- tives that
you aim to achieve in the period of funding. The following points should be
addressed: 